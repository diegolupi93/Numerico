\documentclass{article}
\setlength{\parindent}{0cm}
\usepackage[utf8]{inputenc}
\usepackage{amsmath}
\usepackage{mathrsfs}
\usepackage{amsthm,amssymb}
\renewcommand{\qedsymbol}{\rule{0.7em}{0.7em}}
\begin{document}

\section{Teoremas sin demostracion}\label{Teoremas}


\subsection{Teorema del valor intermedio para funciones continuas}\label{valor-intermedio}

Sea f una funcion continua en [a,b] sea d entre f(a) y f(b) entonces existe $c \in [a,b]$ tal que$f(c)=d $

 \vspace{5mm}

\subsection{Teorema del valor medio}\label{valor-medio}

Sea f una funcion continua en [a,b] y derivable en (a,b). Entonces para cada $x,c \in [a,b]$ se cumple
que $\left( \displaystyle\frac{f(x)-f(c)}{x-c} \right) = f'(\xi) $, para algún $\xi$ entre x y c

 \vspace{5mm}




\subsection{Teorema del Taylor}\label{Taylor}

Sea $f \in C^{(n)}[a,b]$ y existe $f^{(n+1)}$ en (a,b), entonces $\forall x,c \in [a,b]$ se tiene:

$f(x) = \sum\limits_{k=0}^{n} \frac{1}{k!} f^{(k)}(c)(x-c)^k+E_n(x)$ donde

$E_n(x) = \frac{1}{(n+1)!} f^{(n+1)}(\xi)(x-c)^{n+1}$ con $\xi$ entre x y c


 \vspace{5mm}

\subsection{Metodo Newton Global}\label{Metodo-Newton-Global}

Sea una función, tal que f'' continua, f convexa, creciente y tiene un raíz, entoncesraíz es único y el metodo de Newton
converge $ \forall x_0 \in {\rm I\!R} $

 \vspace{5mm}

 \subsection{Unicidad  de conjunto linealmente independiente}\label{Metodo-Newton-Global}

Si $\{ \phi_0(x),...,\phi_n(x) \}$ es un conjunto LI(linealmente independiente) en el espacio de polinomio de
grado $\leq n$, entonces todo polinomio de grado $\leq n$ puede escribirse de manera única como combinación lineal
de $\{ \phi_0(x),...,\phi_n(x) \}$

 \vspace{5mm}

 \subsection{Aproximación por mínimos cuadrados}\label{Metodo-Newton-Global}

Si $\{ \phi_0,...,\phi_n \}$ es un conjunto ortogonal de funciones en [a,b] respecto de w,
entonces la aproximación de cuadrados mínimos a f en [a,b] es

$P(x) = \sum\limits_{k=0}^n a_k\phi_k(x)$, para $k = 0,...,n$, donde

$a_k = \frac{\displaystyle\int_{a}^{b}w(x)f(x)\phi_k(x)dx}{\displaystyle\int_{a}^{b}w(x)(\phi_k(x))^2dx}$

 \vspace{5mm}

 \subsection{Relación de recurrencia}\label{Metodo-Newton-Global}

 El conjunto de funciones polinomiales $\{ \phi_0,...,\phi_n \}$ definido de la siguiente forma ortogonal
 en [a,b], respecto de la función de w(x)

 $\phi_0(x) = 1 $, $\phi_1(x) = x -B_1$, $x \in [a,b]$, donde

 $B_1 = \frac{\displaystyle\int_{a}^{b}w(x)x(\phi_0(x))^2dx}{\displaystyle\int_{a}^{b}w(x)(\phi_0(x))^2dx}$,

  \vspace{5mm}

 Para $k \geq 2$

 $\phi_k(x) = (x-B_k)\phi_{k-1}(x)-C_k\phi_{k-2}(x)$, $x \in [a,b]$, donde

$B_k = \frac{\displaystyle\int_{a}^{b}w(x)x(\phi_{k-1}(x))^2dx}{\displaystyle\int_{a}^{b}w(x)(\phi_{k-1}(x))^2dx}$, y

$C_k = \frac{\displaystyle\int_{a}^{b}w(x)x(\phi_{k-2}(x))^2dx}{\displaystyle\int_{a}^{b}w(x)(\phi_{k-2}(x))^2dx}$


 \vspace{5mm}


 \subsection{Corolario}\label{Metodo-Newton-Global}

Para todo $n > 0$ el conjunto de funciones gemeradas en el teorema de Relación de recurrencias(anterior) es LI en [a,b]
y $\displaystyle\int_{a}^{b}w(x)\phi_n(x)\phi_k(x)dx = 0$ con $k < n$


 \vspace{5mm}





\section{Teoremas con demostracion}

\subsection{Metodo Biseccion (convergencia)}\label{Metodo-Biseccion-(convergencia)}

Si [$a_0,b_0$], [$a_1,b_1$],....., [$a_n,b_n$],.... denotan los sucesivos intervalos del método de bisección, entonces
$ \exists \lim_{n \to \infty} a_n, \lim_{n \to \infty} b_n $ donde son iguales representan una raíz de f.

Si $  C_n =  \left( \displaystyle\frac{a_n+b_n}{2} \right) $ y $r = \lim_{n \to \infty} C_n$ entonces
$\frac{a_n+b_n}{2}$

$| r - C_n | \leq | \frac{1}{2^{(n+1)}}(b_0 - a_0) |$

 \vspace{5mm}


Dem: Si $[a_0,b_0],[a_1,b_1],....$ son los intervalos del algoritmo de bisección. Entonces

(1) $a_0 \leq a_1 \leq ... \leq b_0 $

(2) $b_0 \leq b_1 \leq ... \leq a_0 $

(3) $b_{n+1}-a_{n+1} = \frac{1}{2}(b_n-a_n)$


Como sabemos que $\{ a_n \}$ es acotada superiormente, además es no decreciente, entonces es convergente (1)

Luego como $\{ b_n \}$ es acotada inferiormente, además es no creciente, entonces es convergente (2)

Ademas

$ b_{n+1} - a_{n+1} = \left( \displaystyle\frac{1}{2} \right) (b_{n} - a_{n}) = \left( \displaystyle\frac{1}{2^2} \right) (b_{n-1} - a_{n-1}) 
= ... =\left( \displaystyle\frac{1}{2^{n+1}} \right) (b_0 - a_0) $ 

Entonces

$\lim_{n \to \infty} ({b_{n+1}-a_{n+1}}) = {(\lim_{n \to \infty} b_{n+1}) -(\lim_{n \to \infty}a_{n+1})} $ 

$= \lim_{n \to \infty} \left( \displaystyle\frac{b_0-a_0}{2^{n+1}} \right) = 0$

 \vspace{5mm}

Luego $ {(\lim_{n \to \infty} b_{n+1}) -(\lim_{n \to \infty}a_{n+1})} = 0 $ entonces

$ \lim_{n \to \infty} b_{n+1} = \lim_{n \to \infty} a_{n+1} = r $

 \vspace{5mm}

Veamos que ambos limites tiende a una raíz de f, es decir, veamos que r es una raíz f. Sabemos que f($a_n$)*f($b_n$) $ \leq 0 $

Entonces si tomamos limite, como f es continua, obtenemos que

$ \lim_{n \to \infty} f(a_{n})\lim_{n \to \infty} f(b_{n}) = f(\lim_{n \to \infty} a_{n})f(\lim_{n \to \infty} b_{n}) = f(r)^2 \leq 0 $

Entonces $f(r) = 0$, por ende r es una raíz de f

Veamos que $| r - C_n | \leq | 2^{-(n+1)}(b_0 - a_0) |$

Tenemos que:

$\left| r - C_n \right| \leq \left| \left( \displaystyle\frac{1}{2} \right)(b_{n+1}-a_{n+1}) \right| \leq \left| \left( \displaystyle\frac{1}{2^2} \right)(b_{n}-a_{n}) \right| \leq ... \leq
 \left| \left( \displaystyle\frac{1}{2^{n+1}} \right)(b_0-a_0) \right| $ $\blacksquare$




 \vspace{5mm}


\subsection{Metodo Newton (convergencia)}\label{Metodo-Newton-(convergencia)}

Sea f : ${\rm I\!R} \rightarrow {\rm I\!R}$ una funcion tal que f'' es continua y f'(r) $ \not= 0$ donde r es raíz de f, 
entonces existe $ \delta$ tal que, si el punto inicial de método de Newton $X_0$ satisface $ |r-X_0| \leq \delta $, luego todas las aproximaciones
generadas por el algoritmo$\{X_n\}$ satisfacen $|r-X_n| \leq \delta$, la sucesión $\{X_n\}$ converge a r y la convergencia es cuadrática.

$\left| X_{n+1} - r  \right | \leq C(\delta) \left | X_n -r  \right |^2 $ (convergencia cuadrática)

\vspace{5mm}


Dem: Sea $e_n = r - X_n $(error en la etapa n)

 En la etapa n+1, tenemos: 

 $ e_{n+1} = r - X_{n+1} = r - \left(X_n - \displaystyle\frac{f(X_n)}{f'(X_n)} \right)  = r - X_n +\displaystyle\frac{f(X_n)}{f'(X_n)}
 = e_n + \displaystyle\frac{f(X_n)}{f'(X_n)}$

 $ =  \left( \displaystyle\frac{e_nf'(X_n)+f(X_n)}{f'(X_n)} \right) $(1)

 \vspace{5mm}

 Sabemos por Taylor que f alrededor de $X_n$ tenemos:


 $f(X_n+h)=f(X_n)+f'(X_n)h+\frac{1}{2}f''(\xi_n)h$, luego si tomamos $h=e_n$ obtenemos


 $X_n+h = X_n+e_n = X_n + (r-X_n)$, por ende


 $0 = f(r) = (f(X_n) + f'(X_n)e_n + f''(\xi_n)e_n^2)$, $\xi_n$ entre x y r

 Entonces $f(X_n)+f'(X_n)e_n = -\frac{1}{2}f''(\xi_n)e_n^2 $, (2), de (1) y (2) se obtiene:


 $e_{n+1} = -\frac{1}{2}\frac{f''(\xi_n)}{f'(X_n)}e_n^2$ (3)

\vspace{5mm}

 Para acotar (3), definimos


 $C(\delta) = \frac{1}{2}\left( \displaystyle\frac{\max_{\{|x-r| \leq \delta\}} |f''(x)|}{\min_{\{|x-r| \leq \delta\}}|f'(x)|} \right)$

 \vspace{5mm}

 Como f' y f'' son continuas alrededor de r, luego $|f'(x)|$ y $|f''(x)|$ alcanzan su mínimo y su máximo, respectivamente
 en el intervalo cerrado y acotado $[r-\delta,r+\delta]$. Luego dado $\delta > 0$, para todo x y $\xi$ talque $|x-r \leq \delta|$
 y $|\xi-r| \leq \delta$.


 Se tiene que $\frac{1}{2}\frac{f''(\xi)}{f'(x)} \leq C(\delta)$

 \vspace{5mm}

 Ahora elegimos un $ \delta  $ tan pequeño tal que $ \delta C(\delta) < \rho $

Esto es posible si $\delta \rightarrow 0$, $C(\delta) \rightarrow \frac{1}{2}\frac{f''(r)}{f'(r)}$, bien
definido, pues por hipótesis,

 $f'(r) \not= 0$, por lo tanto, $\delta C(\delta) \rightarrow 0$

 Supongamos que $X_n$ es tal que $|e_n| = |X_n-r| \leq \delta$

 Como $\xi_n$ esta entre $X_n$ y r, $|\xi_n-r| \leq \delta$

 Por def de $C(\delta)$ tenemos $\frac{1}{2}\frac{f''(\xi_n)}{f'(X_n)} \leq C(\delta)$. Luego por (3)

 \vspace{5mm}

 $|e_{n+1}| = \frac{1}{2}\frac{|f''(\xi_n)|}{|f'(X_n)|}e_n^2 \leq C(\delta)e_n^2 = C(\delta)|e_n||e_n| \leq C(\delta)\delta |e_n| \leq \rho|e_n|$

 \vspace{5mm}

 Luego $X_{n+1}-r = |e_{n+1}| \leq \rho|e_n| < |e_n| \leq \delta$

 Por último si $X_0$ es talque $|X_0-r| \leq \delta$, luego por lo anterior $|e_n| \leq \rho|e_{n+1}| \leq ... \leq \rho^n|e_0|$

 Como $0<\rho<1$ y $e_0 \leq \delta$, $\lim_{n \to \infty} f^n = 0$ , $\lim_{n \to \infty} |e_n| = 0$ y $\lim_{n \to \infty} X_n = r$ $\blacksquare$


 

\vspace{10mm}

\subsection{Propiedades de Punto Fijo}\label{Punto}

Sea g continua en [a,b]

(1) si $g(a) \in [a,b]$ y $g(b) \in [a,b]$ entonces existe $r \in [a,b]$ tal que $g(r) = r$  

(2) si ademas existe g' tal que $|g(x)| \leq k$ $\forall x \in (a,b)$ y para algún $k \in (0,1)$, entonces el punto fijo 
es único en (a,b) $\forall x \in (a,b)$

\vspace{5mm}

Dem:

(1) Si $g(a) = a$ ó $g(b) = b$, entonces nada que probar, si esta no existe,

$g(a)>a$ y $g(b)<b$. Definimos $h(x) = g(x)-x$, h continua, en [a,b], tenemos

$h(a) = g(a)-a > 0$, $h(b) = g(b)-b < 0$, por lo tanto por Teorema de Valor Intermedio sabemos que
existe $ r \in (a,b)$ tal que $h(r)=0$, entonces $g(r) = r$

\vspace{5mm}

(2) Supongamos que existen p y q en [a,b] tal que $g(p) = p$, $g(q) = q$, con $p\not=q$

Por Teorema de Valor Medio, $g(p)-g(q) = g'(\xi)(p-q)$, con $\xi$ entre p y q, luego

$|p-q| = |g(p)-g(q)| =|g'(\xi)||(p-q)| \leq k|p-q|<|p-q|$, absurdo, entonces $p=q$$\blacksquare$

\vspace{10mm}

\subsection{Convergenca de Punto Fijo}\label{Punto-f}

Sea g una función tal que $g(x) \in [a,b]$ $\forall x \in [a,b]$, ademas supongamos 
$|g'(x)| < k$ con $0 < k < 1$ $\forall x \in (a,b)$. Entonces para cualquier $p_0 \in [a,b]$,
la sucesión definida por $p_n = g(p_{n-1})$, para $n \geq 1$, converge al unico punto fijo en [a,b]

\vspace{5mm}

Dem:

Por el teorema anterior sabemos que existe punto fijo p en [a,b]. Como la g transforma [a,b] en si mismo,
la sucesión $\{ p_n \}_n$ esta bien definida $\forall n \geq 0$ y $p_n \in [a,b] \forall n$.

Veamos la convergencia:

$p_n-p= g(p_{n-1})-g(p) = |g'(\xi_n)||p_{n-1}-p| \leq k|p_{n-1}-p|$, luego

$|p_n-p| \leq k|p_{n-1}-p| \leq .... \leq k^n|p_0-p|$, como $ 0 < k < 1$, entonces

$\lim_{n \to \infty} |p_n-p| = 0$ $\blacksquare$

\vspace{10mm}


\subsection{Unicidad Polinomio Interpolante}\label{Unicidad-Polinomio-Interpolante}

Sean $ x_0,...,x_n $ reales tal que $x_0<...<x_n$ con $y_0,...,y_n$ arbitrarias asociadas, entoces
existe un único polinomio $P(x)$ tal que $gr(P) \leq n $ que interpola a los puntos $x_0,...,x_n$, es decir
$P(x_i) = y_i$, con $ i = 0,...,n$

\vspace{5mm}

Dem:

(1) Interpolación

Veamos unicidad, para ello supongamos que existen dos polinomios de grado $\leq n$ tal que $P_n(x_i) = y_i$ y 
$Q_n(x_i) = y_i$, para $i=0,...,n$. Sea 

\vspace{5mm}

$h_n(x) = P_n(x)-Q_n(x) $, luego es un polinomio de grado $\leq n$

Luego se observa que $h_n(x) = P_n(x)-Q_n(x) = 0$ con $i=0,...,n$, pero como $h_n$ es un polinomio con (n+1) raíces,
entonces

 $h_n(x) = 0 $ $\forall x$, por lo tanto $P_n(x) = Q_n(x) $ $\forall x$

\vspace{5mm}

Veamos existencia, vamos a demostrar su existencia mediante el metodo de lagrange y Newton, veamos primero el metodo de lagrange.

$\ell_i(x) = \left( \displaystyle\frac{(x-x_0)(x-x_1)...(x-x_{i-1})(x-x_{i+1})...(x-x_n)}{(x_i-x_0)(x_i-x_1)...(x_i-x_{i-1})(x_i-x_{i+1}...(x_i-x_n)} \right)  $

%$ = \prod_{j=0  \and j \not= i}^{n} \left( \displaystyle\frac{(x-x_i)}{(x_i-x_j)} \right)  $
 $ = \prod\limits_{\substack{j=0 \\ j\neq i}}^{n} \left( \displaystyle\frac{(x-x_i)}{(x_i-x_j)} \right) $

 Como sabemos que

 $$
\ell_i(x_j)=
\begin{cases}
1 & j=i\\
0 & j\not=i
\end{cases}
$$

Luego  $P(x) = \sum\limits_{i=0}^n Y_i\ell_i(x)$

Por lo tanto $P(x_i) =\sum\limits_{i=0}^n Y_i\ell_i(x_i) = Y_i $

$\blacksquare$





\vspace{10mm}

\subsection{Error Polinomio Interpolante}\label{Error-Polinomio-Interpolante}

Sea  $f \in C^{n+1}(a,b)$ y P un polinomio de grado$\leq n$ que interpola a f en (n+1) puntos distintos 
$x_0,...,x_n$ en [a,b]. Entonces para cada $x \in [a,b]$ existe $\xi \in (a,b)$ tal que

$f(X)-P(x)= \displaystyle\frac{1}{(n+1)!} f^{(n+1)}(\xi) \prod\limits_{\substack{i=0}}^{n} ( x-x_i)$

Dem:

Si $x=x_i$

$0 = f(x_i)-P(x_i) = \displaystyle\frac{1}{(n+1)!} f^{(n+1)}\xi\prod\limits_{\substack{j=0}}^{n} ( x_i-x_j) = 0$, ya que $i \in \{0,...,j,...,n\}$, luego vale para $x=x_i$


Si $x\not=x_i$

Sea $w(t)=\prod\limits_{\substack{i=0}}^{n} (t-x_i)$

$C = \frac{f(x)-p(x)}{w(x)}$, (constante, con $w(t) \not= 0$)

$\phi(t) = f(t)-P(t) - Cw(t)$ (función en t), Por lo tanto

Luego como $\phi(x_i) = 0$ para cada $i\in\{ 0,...,n \}$ y 

$\phi(x) = f(x)-P(x)-Cw(x) = f(x) - P(x) - \frac{f(x) - P(x)}{w(x)}$, luego 

\vspace{5mm}

$\phi'$ tiene al menos n+1 raices en [a,b]

$\phi''$ tiene al menos n raices en [a,b]

.

.

.

$\phi^{(n+1)}$ tiene al menos una raíz en [a,b]

\vspace{5mm}

Entonces sea $\xi$ esa raíz de $\phi^{(n+1)}, \xi \in (a,b)$

Asi

$(0 = \phi^{(n+1)}(\xi) = f^{(n+1)}(\xi)-P^{(n+1)}(\xi) - Cw^{(n+1)}(\xi))$ (*)

\vspace{5mm}

Como $w(t)=\prod\limits_{\substack{i=0}}^{n} (t-x_i)$, entonces $ w(t) = t^{(n+1)} + Q(x) $ $gr(Q) \leq n$

Luego $w(t)^{(n+1)}=(n+1)!$, entonces $w^{(n+1)}(\xi) = (n+1)!$

\vspace{5mm}

Finalmente de (*)

$0 = f^{(n+1)}(\xi)- C(n+1)! = f^{(n+1)}(\xi) - \frac{f(x) - P(x)}{w(x)}(n+1)! = 0$

$f(x) - P(x) = \frac{f^{(n+1)}(\xi)}{(n+1)!}\prod\limits_{\substack{i=0}}^{n} ( x-x_i)$
$\blacksquare$

\vspace{10mm}

\subsection{Diferencias Divididas}\label{Relacion-recursiva-de-Diferencias-Divididas}

Las diferencias divididas satisfacen la ecuación $f[x_0,...,x_n] = \frac{f[x_1,...,x_n] - f[x_0,...,x_{n-1}]}{x_n - x_0}$

\vspace{5mm}

Dem:

Sea $P_{n-1}$ el polinomio que interpola a f en $x_0,...,x_{n-1}$, con $gr(P_{n-1}) \leq n-1$

Sea Q el polinomio que interpola a f en $x_1,...,x_n$, con $gr(Q) \leq n-1$

Sea $P_n$ el polinomio dado por $P_n(x) = (Q(x)+\frac{x-x_n}{x_n-x_0}(Q(x) - P_{n-1}(x))$)(*)

\vspace{5mm}

Veamos que $P_n$ interpola a f en $x_0,...,x_n$, con $gr(P_n) \leq n$

Para $i = 1,...,n-1$

$P_n(x_i) = f(x_i)$, ya que $Q(x_i) + \frac{x_i-x_n}{x_n-x_0}(Q(x_i)-P_{n-1}(x_i))$,

como $Q(x_i) = f(x_i) = P_{n-1}(x_i)$, entonces $P_n = Q(x_i)$

\vspace{5mm}

Si $i = 0$

$P_n(x_0) = f(x_0)$, ya que $Q(x_0)+\frac{x_0-x_n}{x_n-x_0}(Q(x_0)-P_{n-1}(x_0))$

Como $\frac{x_0-x_n}{x_n-x_0} = -1$, entonces

$Q(x_0)+\frac{x_0-x_n}{x_n-x_0}(Q(x_0)-P_{n-1}(x_0)) = P_{n-1}(x_0) = f(x_0)$  

\vspace{5mm}

Si $i = n$

$P_n(x_n) = f(x_n)$, ya que $Q(x_n)+\frac{x_n-x_n}{x_n-x_0}(Q(x_n)-P_{n-1}(x_n))$

Como $\frac{x_n-x_n}{x_n-x_0} = 0$, entonces

$Q(x_n)+\frac{x_n-x_n}{x_n-x_0}(Q(x_n)-P_{n-1}(x_n)) = Q(x_n) = f(x_n)$ 

\vspace{5mm}

Luego $P_n$ y (*) son pol de grado $\leq n$ que interpola a f en los (n+1) puntos $x_0,...,x_n$,
por unicidad del polinomio interpolante, $P_n$ y (*) son los mismos polinomios.

\vspace{5mm}

Veamos cual es el coeficiente de $x^n$, como

$P_n(x) = Q(x)+\frac{x-x_n}{x_n-x_0}(Q(x) - P_{n-1}(x))$, tenemos

$f[x_0,...,x_n] = \frac{f[x_1,...,x_n] - f[x_0,...,x_{n-1}]}{x_n - x_0}$


$\blacksquare$


\vspace{10mm}




\subsection{Punto que no pertenece a los puntos de interpolación}\label{Permutacion-De-Dif-Divididas}

Sea P el polinomio de grado $\leq n$ que interpola f en los (n+1) nodos $x_0,...,x_n$(distintos). Si t es
un punto distinto de los nodos, entonces 

$f(t) - P(t) = f[x_0,...,x_n,t]\prod\limits_{\substack{i=0}}^{n} (t-x_i)$


\vspace{5mm}

Dem:
Sea Qel polinomio de grado $\leq n+1$ que interpola a f en los pintos $x_0,...,x_n,t$.Entoces

$Q(x) = P(x) + f[x_0,...,x_n,t]\prod\limits_{\substack{i=0}}^{n} (t-x_i)$

Como $Q(t) = f(t)$, obtenemos $f(t) - P(t) = f[x_0,...,x_n,t]\prod\limits_{\substack{i=0}}^{n} (t-x_i)$$\blacksquare$

\vspace{10mm}



\subsection{Relacion Diferencias Divididas con derivada n-ésima}\label{Relacion-DIf-Divididas-con-derivada-n-esima}

Si f es n veces continuamente diferenciable en (a,b) y $x_0,...,x_n$ nodos distintos en [a,b], entonces existe
$\xi \in (a,b)$ tal que

$f[x_0,...,x_n] = \displaystyle\frac{f^{(n)}(\xi)}{n!}$

\vspace{5mm}

Dem:

Sea P el polinomio de grado $\leq n-1$ que interpola a f en $x_0,...,x_{n-1}$.

Por el teorema del error en el polinomio interpolante tenemos

$f(x_n)-P(x_n) = \displaystyle\frac{f^{(n)}(\xi)}{n!} \prod\limits_{\substack{i=0}}^{n-1} (x_n-x_i)$ (A)

\vspace{5mm}

Por el teorema anterior

$f(x_n)-P(x_n) = f[x_0,...,x_n]\prod\limits_{\substack{i=0}}^{n-1} (x_n-x_i)$ (B)

\vspace{5mm}

Por lo tanto de (A) y (B) se obtiene $f[x_0,...,x_n] = \displaystyle\frac{f^{(n)}(\xi)}{n!}$

 $\blacksquare$
 

\vspace{10mm}



\subsection{Error en la interpolacion en un punto}\label{Suavidad-Spline-cubico}

Sea f definida en [a,b], n veces continuamente derivable en [a,b]. Sean $x_0,...,x_n \in [a,b]$ distintos, $y \in [a,b]$. Entonces

$\lim_{{(x_0,...,x_n)} \to {(y,...,y)}}f[x_0,...,x_n] = \frac{f^{(n)}(y)}{n!}$

\vspace{5mm}

Dem:

Sabemos que $\exists \xi \in (a,b)$ tal que

$f[x_0,...,x_n] = \frac{f^{(n)}(\xi)}{n!}$

Por ende si $(x_0,...,x_n) \to (y,...,y)$, entonces $\xi \to y$

\vspace{5mm}

Luego tomamos el limite:

$\lim_{{(x_0,...,x_n)} \to {(y,...,y)}}f[x_0,...,x_n] = \lim_{{(x_0,...,x_n)} \to {(y,...,y)}}\frac{f^{(n)}(\xi)}{n!} = \frac{f^{(n)}(y)}{n!} $

$\blacksquare$

\vspace{10mm}

\subsection{Linealmente Indenpendiente}\label{LI}

Si $\phi_j$ es un polinomio de grado j, $j = 0,...,n$, entonces $\{\phi_0,...,\phi_n\}$ es LI en cualquier intervalo [a,b]

\vspace{5mm}

Dem:

Sean $C_0,...,C_n \in {\rm I\!R}$ tal que $P(x) = C_0\phi_0(x)+...+C_n\phi_n(x) = 0$, para cada $x \in [a,b]$.
Queremos ver que $C_j = 0$, para $j = 0,...,n$. Como $P(x)$ se anula para cada $x \in [a,b]$, los coeficientes
de cada potencia de x debe ser cero. En particular, el coeficiente de $x^n$ es cero. Como el único término que 
$x^n$ es $C_n\phi_n(x)$, entonces $C_n = 0$. Luego

$P(x) = C_0\phi_0(x)+...+C_{n-1}\phi_{n-1}(x)$. Repitiendo esto (n-1) veces obtenemos

$C_1 = ... = C_n = 0$




\vspace{10mm}



\subsection{$A^tAx=A^tb\Leftrightarrow$  x minimiza cuadrados minimos}\label{Relacion-DIf-Divididas-con-derivada-n-esima}

Sea $A \in  {\rm I\!R}^{nxm}$, $x \in {\rm I\!R}^{m}$ con $m \leq n$ es solución del problema de cuadrados mínimos si y solo si

$A^TAx = A^Tb$ para algun $b \in {\rm I\!R}^{m}$, Ademas si A tiene rango pmpleto, la solución x es única.

Dem:

Probemos primero $\Rightarrow$)

Si x es solución del problema de cuadrados minimos debemos ver que resuelve el sistema $A^TAx = A^Tb$, para algún b

Por hipotesis $||b - Ax||^2 \leq ||b-Ay||^2$, para todo$y \in {\rm I\!R}^m$

Sea $y = x+ tz$ con $z \in {\rm I\!R}^m$, entonces

\vspace{5mm}

$||b - Ax||^2 \leq ||b-Ay||^2 =||b-A(x+ tz))||^2 = ||b- Ax - Atz))||^2 $, entonces obtenemos

\vspace{5mm}

$0 \leq -2t\langle b-Ax, Az \rangle + t^2||Az||^2$

\vspace{5mm}

$2t\langle b-Ax, Az \rangle \leq t^2||Az||^2 $, ahora

\vspace{5mm}

Si $t > 0$

$2\langle b-Ax, Az \rangle \leq t||Az||^2$

si $t<0$

$t||Az||^2 \leq 2\langle b-Ax, Az \rangle$, por ende si tomo el limite de ambos lados

\vspace{5mm}

$\langle b-Ax, Az \rangle = 0$, luego

$0 = (Az)^T(b-Ax)=z^TA^T(b-Ax)=z^T(A^Tb-A^TAx)$, entonces

$A^T-A^TAx=0$, por lo tanto $A^TAx=A^Tb$

\vspace{5mm}

$\Leftarrow$)Probemos que si $\bar{x}$ es solución de $A^TAx=A^Tb$ entonces es minimizador de $||b-Ax||$

Quiero ver que $||b-A\bar{x}||^2 \leq ||b-Ax||$, para todo $x \in {\rm I\!R}^m$

\vspace{5mm}

$||Ax-b||^2= ||Ax-A\bar{x}+A\bar{x}-b||^2=||Ax-A\bar{x}||^2+2\langle Ax-A\bar{x}, A\bar{x}-b \rangle+||A\bar{x}-b||^2$

\vspace{5mm}

Veamos que $0 \leq \langle Ax-A\bar{x}, A\bar{x}-b \rangle$

\vspace{5mm}

$\langle A(x-\bar{x}), A\bar{x}-b \rangle = \langle A\bar{x}-b , A(x-\bar{x})\rangle =  (A(x-\bar{x}))^T (A\bar{x}-b) $

\vspace{5mm}

$(x-\bar{x})^TA^T (A\bar{x}-b) = (x-\bar{x})^T (A^TA\bar{x}-A^Tb) = 0$, por lo tanto

\vspace{5mm}

$||A\bar{x}||^2 \leq ||Ax-b||^2$, para todo $x \in {\rm I\!R}^m$ $\blacksquare$


\vspace{10mm}

\subsection{$A^tAx=A^tb$unica solucion$\Leftrightarrow$  A rango completo}\label{Relacion-DIf-Divididas-con-derivada-n-esima}

Preguntar$\blacksquare$

\vspace{10mm}

\subsection{Error trapecio integracion numerica}\label{Error-trapecio-integracion-numerica}

Error trapecio $= - f''(\xi)\displaystyle\frac{(b-a)^3}{12}$

Dem:

(1) Error de interpolación

(2)Teorema de valor intermedio para integrales

$f(x)-P(x) =^{(1)} \displaystyle\frac{1}{2!}f''(\xi_x)(x-a)(x-b) $, donde $\xi_x \in [a,b]$

Ahora

$\displaystyle\int_{a}^{b}f(x)-P(x)dx =  \displaystyle\frac{1}{2!}\displaystyle\int_{a}^{b}f''(\xi_x)(x-a)(x-b)dx$

$=^{(2)}\displaystyle\frac{1}{2!}f''(\xi_x)\displaystyle\int_{a}^{b}(x-a)(x-b)dx
=-f''(\xi)\displaystyle\frac{(b-a)^3}{12}$ $\blacksquare$


\vspace{10mm}

\subsection{Error simpson integracion numerica}\label{Error-simpson-integracion-numerica}

Error simpson $= - \displaystyle\frac{(b-a)^5}{90} f^{(4)}(\xi)$

Dem:

El termino de error en la regla de simpson se puede establecer usando la serie de Taylor

$f(a+h) = f + hf' + \displaystyle\frac{1}{2!}h^2f''+... $, luego por sustitucion obtenemos

\vspace{5mm}

$f(a+2h) = f + 2hf' + \displaystyle\frac{1}{2!}2h^2f''+...$, con estas 2 series se obtiene

\vspace{5mm}

$f(a)+4f(a+h)+f(a+2h)=6f+6hf'+4h^2f''+...$, y asi tenemos

\vspace{5mm}

$\displaystyle\frac{h}{3}[f(a)+4f(a+h)+f(a+2h)]=2hf+2h^2f'+\displaystyle\frac{4}{3}h^3f''+...$, luego por serie de Taylor

\vspace{5mm}

$F(a+2h) = F(a)+2hF'(a)+2h^2F''(a)+...$

\vspace{5mm}

Sea $F(x) = \displaystyle\int_{a}^{x}f(t)dt$, por el teorema fundamenta del calculo, $F'=f$, obtenemos

\vspace{5mm}

$\displaystyle\int_{a}^{a+2h}f(x)dx=2hf+2h^2f'+\displaystyle\frac{4}{3}h^3f''+...$, luego

\vspace{5mm}

$\displaystyle\int_{a}^{b}f(x)dx \approx \displaystyle\frac{(b-a)}{6}\left[f(a)+4f\left(\displaystyle\frac{a+b}{2} \right)+f(b) \right]$, con el termino de error

\vspace{5mm}

$-\displaystyle\frac{1}{90} \left( \displaystyle\frac{b-a}{2} \right)^5 f^{(4)}(\xi)$, para algun $\xi$ en (a,b), ya que

\vspace{5mm}

$\mathcal{O}(h^5) = - \displaystyle\frac{h^5}{90}f^{(4)}(\xi)$$\blacksquare$




\vspace{10mm}

\subsection{Teorema Regla Cuadratura gaussiana}\label{T.-Auxiliar-Cuadratura-gaussiana}

Sea $w(x)>0$,$q(x)$ un polinomio distinto de cero yde grado n+1 que es ortogonal a $\prod_{n}$, esto es

$\displaystyle\int_{a}^{b}q(x)p(x)w(x)dx = 0$, para todo $p \in \prod_{n}$,

si $x_0,...,x_n$ son los ceros de $q(x)$, entonces $\displaystyle\int_{a}^{b}f(x)w(x)dx$ es exacto para $\prod_{2n+1}$

Dem:

Sea $f \in \prod_{2n+1}$, con

\vspace{5mm}

$f \equiv qP+r$, con $P,r \in \prod_{2n+1}$, luego

\vspace{5mm}

$f(x_i) = q(x_i)P(x_i)+r(x_i)=r(x_i)$, ya que $q(x_i)=0$

\vspace{5mm}

Por lo tanto

$\displaystyle\int_{a}^{b}f(x)w(x)dx=\displaystyle\int_{a}^{b}(q(x)P(x)+r(x))w(x)dx=\displaystyle\int_{a}^{b}r(x)w(x)dx=\sum\limits_{i=0}^{n}r(x_i)A_i=\sum\limits_{i=0}^{n}f(x_i)A_i$$\blacksquare$

\vspace{10mm}





\subsection{Teorema de numeros de cambio de signo int numerica}\label{T.-Auxiliar-Cuadratura-gaussiana}

Sea $w(x)>0$, sea f distinta de cero y continua en $a=t_0<...<t_n=b$, donde P es w-ortogonal a $\prod_{n}$, entonces
f cambia de signo n+1 veces en el (a,b)

Dem:

Veamos que existe un cambio de signo en el (a,b), sabemos que $1 \in \prod_{n}$, luego

$\displaystyle\int_{a}^{b} f(x)w(x)dx = 0$, entonces f tiene al menos un cero en (a,b).

Ahora veamos que hay n+1 cambios de signo

Supongamos que hay $r \leq n$ cambios de signo y sean $t_0,...,t_{r+1}$ donde ocurre un cambio de signo, luego

f tiene un signo $[t_0,t_1)$

f tiene un signo $(t_1,t_2)$

.

.

.

f tiene un signo $[t_r,t_{r+1})$

Ahora, sea

$P(x)= \prod\limits_{i=1}^{r}(x-t_i) $, entoces $gr(P) \leq r \leq n$

Como P(x) cambia de signo en los mismos intervalos que f, tiene los mismos signos, luego

$\displaystyle\int_{a}^{b} f(x)P(x)w(x)dx \not= 0$, absurdo, pues f es ortogonal a P por hipotesis

por lo tanto $\displaystyle\int_{a}^{b}f(x)P(x)w(x)dx = 0$, por lo tanto $n+1 \leq r$$\blacksquare$


\vspace{10mm}





\subsection{Convergencia de Gauss Seidel}\label{Convergencia-de-Gauss-Seidel}

Si A tiene dominio diagonal, entonces los metodos  de Gauss-Seidel converge

Dem:

Sea $\delta(I-Q^{-1}A)<1$, sea x un autovector de autovalor $\lambda$ y $||x||_\infty=1$, entonces

$(I-Q^{-1}A)x = \lambda x$ ó $Qx-Ax = \lambda Qx$, luego

$-Ux = \lambda Qx$

$-\sum\limits_{j \not= i+1}^{n}a_{ij}x_j = \lambda \sum\limits_{j=1}^{i}a_{ij}x_j = \lambda a_{ii}+\lambda \sum\limits_{j=1}^{i-1}a_{ij}x_j$, con $1 \leq i \leq n$

$\lambda a_{ii}x_i = -\sum\limits_{j=i+1}^{n}a_{ij}x_j-\lambda\sum\limits_{j=1}^{i-1}a_{ij}x_j$

Sea k tal que $|x_\ell| \leq |x_k|=1$, para todo $\ell$, luego sea $i=k$

\vspace{5mm}

$|\lambda||a_{kk}| \leq \sum\limits|a_{kj}|+|\lambda|\sum\limits_{j=1}^{k-1}|a_{kj}|$, por lo tanto

$|\lambda| \leq \displaystyle\frac{\sum\limits_{j=k+1}^{n}|a_{kj}|}{(|a_{kk}|-\sum\limits_{j=1}^{k-1}|a_{kj}|)} < 1$

Entonces $\delta(I-Q^{-1}A)<1$, veamos que Gauss-Seidel converge

\vspace{5mm}

$|a_{kk}| > \sum\limits_{j<k}|a_{kj}|+\sum\limits_{j>k}|a_{kj}|$, luego

\vspace{5mm}

$|a_{kk}|-\sum\limits_{j<k}|a_{kj}| > \sum\limits_{j>k}|a_{kj}|$

\vspace{5mm}

$1 > \displaystyle\frac{ \sum\limits_{j>k}|a_{kj}|}{|a_{kk}|-\sum\limits_{j<k}|a_{kj}|} $, por lo tanto converge$\blacksquare$



\vspace{10mm}

\end{document}
