\documentclass{article}
\setlength{\parindent}{0cm}
\usepackage[utf8]{inputenc}
\usepackage{amsmath}
\usepackage{mathrsfs}
\usepackage{amsthm,amssymb}
\renewcommand{\qedsymbol}{\rule{0.7em}{0.7em}}
\begin{document}

\section{Practico 1}\label{Practico-1}


\subsection{Taylor}\label{Taylor}

$f(x) = \sum\limits_{k=0}^{n} \frac{1}{k!} f^{(k)}(c)(x-c)^k+E_n(x)$ donde

$E_n(x) = \frac{1}{(n+1)!} f^{(n+1)}(\xi)(x-c)^{n+1}$ con $\xi$ entre x y c

 \vspace{5mm}
\subsection{Error}\label{Error}

Sean $\{ X_n \}$ y $\{ \alpha_n \}$, dos sucesiones distintas decimos que:

$X_n = \mathcal{O}(\alpha_n)$, si existe $C > 0$ y $r \in \mathbb{N}$ tal que

$|X_n| \leq C|\alpha_n|$, $\forall n \geq r$

 \vspace{5mm}

Decimos $f(x) = o(g(x)) \Longleftrightarrow \lim_{x \to m} \left|\frac{f(x)}{g(x)} \right| = 0$
 \vspace{5mm}

 Error absoluto = $\frac{|real-aproximado|}{|real|}$

 Error relativo = $ |real - aproximado| $

  \vspace{5mm}

\section{Practico 2}\label{Practico-2}

\subsection{Bisección}\label{Biseccion}

$I = [x_0,x_1] \rightarrow a_0 = x_0$ y $b_0 = x_1$, ademas $c_0 =  \frac{a_0+b_0}{2}$, para poder interar debe cumplir
$f(a_0)f(b_0) < 0$

Algoritmo, sea $a = x_0$, $b = x_1$ y $c = \frac{a+b}{2}$

Paso 1: Si $f(c) = 0 $, terminar, y c es la raíz, Si no, ir a Paso 2

Paso 2: Si $signo(f(a)) = signo(f(c))$, entonces $a = c$ y $b = b$, si no $a = a$ y $b = c$ y voy a Paso 1

  \vspace{5mm}

Error método de Bisección

$|r - c_0| \leq \frac{|b_0 - a_0|}{2}$, entonces $|r-c_n| \leq \frac{|b_n - a_n|}{2}$, por ende $|r-c_n| \leq \frac{|b_0 - a_0|}{2^{(n+1)}}$ 

\vspace{5mm}

\subsection{Newton}\label{Newton}

Sea f derivable entonces, dado un $x_0$ obtenemos la siguiente sucesión

$X_{n+1} = X_n - \frac{f(X_n)}{f'(X_n)}$

Error del metodo

$e_{n+1} = \frac{e_nf'(X_n)+f(X_n)}{f'(X_n)}$


\vspace{5mm}







\end{document}